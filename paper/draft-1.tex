\documentclass[twoside]{article}

\usepackage{lipsum} % Package to generate dummy text throughout this template

\usepackage[T1]{fontenc} % Use 8-bit encoding that has 256 glyphs
\linespread{1.05} % Line spacing - Palatino needs more space between lines
\usepackage{microtype} % Slightly tweak font spacing for aesthetics

\usepackage[hang, small,labelfont=bf,up,textfont=it,up]{caption} % Custom captions under/above floats in tables or figures
\usepackage{booktabs} % Horizontal rules in tables
\usepackage{hyperref} % For hyperlinks in the PDF

\usepackage{fullpage}

\usepackage{titlesec} % Allows customization of titles
\renewcommand\thesection{\Roman{section}} % Roman numerals for the sections
\renewcommand\thesubsection{\Roman{subsection}} % Roman numerals for subsections
\titleformat{\section}[block]{\large\scshape\centering}{\thesection.}{1em}{} % Change the look of the section titles
\titleformat{\subsection}[block]{\large}{\thesubsection.}{1em}{} % Change the look of the section titles

%----------------------------------------------------------------------------------------
% TITLE SECTION
%----------------------------------------------------------------------------------------

\title{\vspace{-15mm}\fontsize{24pt}{10pt}\selectfont\textbf{Article Title}} % Article title

\author{
\large
\textsc{Miguel Sanchez Jeff Hsuing Kyle Zhang}\\
\normalsize migchez@umich.edu test@umich.edu % Your email address
\vspace{-5mm}
}
\date{}

%----------------------------------------------------------------------------------------

\begin{document}

\maketitle % Insert title

%----------------------------------------------------------------------------------------
% ABSTRACT
%----------------------------------------------------------------------------------------

\begin{abstract}

\noindent \lipsum[1] % Forex, or the Foreign Exchange Market is a market for the trading of currencies from around the world. The system is highly complex and volatile, and thus, even though massive amounts of data on the market are freely available, prediction of the daily trends has been a difficult problem. By modeling Forex prediction as a simple binary classification problem (currency will increase/decrease in value relative to another), we are hoping to be able to leverage modern ML techniques and open sourced libraries in order to predict daily Forex trends with a high level of accuracy. We trained a Neural Network and implemented a Gaussian Process to predict different future foreign exchange values. Then based upon these different model values, we can deduce a predicted trend of currency movement.

\end{abstract}

%----------------------------------------------------------------------------------------
% ARTICLE CONTENTS
%----------------------------------------------------------------------------------------

\section{Introduction}

This chapter presents some background about the foreign exchange market and the reasons why our team is tackling this problem. First, the concepts of currency exchange rates and foreign exchange markets are explained. Followed by the motivation behind this project. 
Currency exchange is the trading of one currency against another and usually takes place as follows. Consumers typically come into contact with currency exchange when they have to convert their current currency in to the currency of a foreign country they intend to travel to. Businesses typically have to deal with currency exchange when they conduct trade outside their home country. Investors and speculators also trade currencies directly in order to benefit from the movements in the currency exchange rate. For example, if an American investor believes that the Swiss economy is strengthening, and as a result expects the Swiss Franc to appreciate in value relative to the U.S. Dollar, the investor may want to buy Swiss Francs and take what is referred to as a long position. Similarly, if the Swiss Franc is believed to go down, the investor may want to sell Swiss Francs against U.S. Dollars, thus taking a short position. Those transactions are often conducted to take advantage of movements in the market over very short time periods. Commercial and investment banks also participate in the currency market, as well as governments and central banks when they try to intervene for adjusting economic imbalances. Currency exchange rates are always quoted for a currency pair using ISO code abbreviations[1]. For example, JPY/US refers to the two currencies US Dollar and Japanese Yen. The first is the base currency and the second the quote currency. 
Therefore, JPY/US specifies how many Japanese Yen you get for one US Dollar. 
The currency exchange market, also referred to as the Forex market, was established in 1971, when floating exchange rates began to materialize. Iwn terms of trading volume, Forex is the world’s largest market, with daily trading volumes in excess of 4 trillion USD[1]. It is by far the most efficient and easily accessible market. Today, an overwhelming majority of all Forex transactions involve a few major currencies, the US Dollar (USD), Euro (EUR), Japanese Yen (JPY), Swiss Franc (CHF), and British Pound (GBP). While there are other currencies in the world, we will only be interested in USD and JPY. The Forex market is open for business around the clock 24/7. 
The reason why we want to predict the exchange rates in the Forex is due to the availability of data and the potential financial and social gains. As we have stated before, there are massive amounts of data recorded for this problem. This is due to the accounting principles of the world in order to audit trades of currencies in order to safeguard against fraudulent practices in the market. Due to these accounting principles, there is data available at such a minute scale that it becomes ineffective. In the previous paragraph, we discussed that there are several factors that can influence the exchange rates. Most countries have detailed historical data on several economic benchmarks which we can use for our problem. 
The incredible range of data available to us makes this problem attractive to solve. 
As for the potential financial and social gains, if we can predict the increase/decrease of a currency, there is obvious financial gains in anticipating exchange rate changes. For example, investing in a currency that will rise in value relative to another, and then exchanging the currency after the rise has occurred will allow an investor to increase their assets. The less obvious social benefit to creating a more accurate foreign exchange rate predictor lies in the fundamentals of the market. Due to the effects of information asymmetry, where people don’t have the same information when making trades, some people can take advantage of other’s ignorance and benefit at the cost to others. In practice, investors with more resources, such as the institutional firms, are more informed than the average investor and thus can take advantage of that fact. So if we can create a predictive model that works, we can level the information playing field for the average investor and and save them the costs of information asymmetry. 
However, if we do somehow engineer a predictor of exchange rates with high enough accuracy on historical data, there is no guarantee that it will work on future/current data. This is because of the anticipatory behavior of participants in the market. This behavior is when participants execute actions that benefit themselves based upon information or beliefs on what the market will do. So if we engineered a perfect predictor, it will immediately become redundant as all participants in the market will trade on this data and immediately remove all opportunities the information can create and exacerbate the volatility in the market. 

%------------------------------------------------

\section{Proposed method}

How are you going to solve this problem? Why should the proposed method work? Provide technical details of your approach if you are proposing a novel method.\\

There has been much research into how various factors influence the exchange rate. Additionally, there is much debate over which factors are the causation of changes in exchange rate and which factors are merely correlated. From our initial research into related work, we have found that there are some widely­used and accepted factors and these are the ones that we will use as features [3,4,5]. They are the inflation rates of countries, money supply of the currency, current accounts (trade balance), GDP, interest rate, and stock indices. After gaining access to several databases such as Bloomberg and Factset, we have acquired the relevant information but each of these features are a time series on some frequency that varies. So, after joining the data to a common frequency, we can now use these as our data for our machine learning models. Another transformation we are doing to the data is to take the log of exponential growth metrics such as GDP, current accounts, money supply to give us a linear pattern to data.  After engineering our features with these two transformations, we have a usable dataset for the problem. 
\lipsum[4] % Dummy text

%------------------------------------------------

\section{Related work}

What are existing methods? What are the state-of-the-art methods for this problem? How is your approach different from the related work?\\

\lipsum[5] % Dummy text

%------------------------------------------------

\section{Experimental results}

Milestones achieved so far (add all relevant experimental results). How do these results support your claim?\\

\lipsum[5] % Dummy text


%------------------------------------------------

\section{Conclusion}

Summary of your results. What have you learned? What’s the biggest contribution of this project?\\

\lipsum[5] % Dummy text


%----------------------------------------------------------------------------------------
% REFERENCE LIST
%----------------------------------------------------------------------------------------

\begin{thebibliography}{99} % Bibliography - this is intentionally simple in this template

\bibitem[Figueredo and Wolf, 2009]{Figueredo:2009dg}
Figueredo, A.~J. and Wolf, P. S.~A. (2009).
\newblock Assortative pairing and life history strategy - a cross-cultural
  study.
\newblock {\em Human Nature}, 20:317--330.
 
\end{thebibliography}

%----------------------------------------------------------------------------------------


\end{document}
